%%
%% Beginning of file 'sample.tex'
%%
%%  2007 September
%%
%% This is a sample manuscript marked up using the
%% AASTeX v5.x LaTeX 2e macros.
%%  mimicing ASTR journal style
% mohamed commands
\documentclass[11pt]{aastex}
\usepackage{spr-astr-addons}
\usepackage{graphicx}
\usepackage{latexsym}
%**************************************************

%\documentclass{aastex}
%\usepackage{spr-astr-addons}
%\usepackage{url}\urlstyle{rm}


\RequirePackage{color}
\def\imagei{\centerline{\color[gray]{.75}\rule{\hsize}{4pc}}}%
\def\imageii{\centerline{\color[gray]{.75}\rule{4pc}{4pc}}}%

\newcommand{\vdag}{(v)^\dagger}
\newcommand\emaila{}


\begin{document}

\title{Continuous Monitoring of Comet Holmes from Before the 2007 Outburst}
\slugcomment{Not to appear in Nonlearned J., 45.}
%% Running heads
\shorttitle{Monitoring of Comet Holmes' 2007 outburst}
\shortauthors{Ehab et al.}

\author{Ehab E. El-Houssieny\altaffilmark{1}}
\and \author{Robert J. Nemiroff\altaffilmark{1}}
\affil{Michigan Technological University, Department of Physics,
1400 Townsend Drive, Houghton, MI 49931, USA}
%??\emaila{nemiroff@mtu.edu}


\and
\author{Timothy E. Pickering\altaffilmark{2}}
%??\emaila{tim@mmto.org}
\affil{MMT Observatory, University of Arizona, Tucson, AZ 85721,
USA}



%%\altaffiltext{1}{First Alternate Affilation.}
%%\altaffiltext{2}{Second Alternate Affilation.}
%%\altaffiltext{3}{Third Alternate Affilation.}


%%%%%%%%%%%%%%%%%%%%%%%%%%%%%%%%%%%%%%%%%%%%%%%%%%%%%%%%%%%%%%%%%%%%%%%%
\begin{abstract}

The outburst and subsequent brightness evolution of Comet Holmes
has been observed using the MMT Observatory's All-Sky Camera
(Pickering 2006) on Mt. Hopkins near Tucson, Arizona, USA.   The
comet was picked up at the limiting visual magnitude of 5.5 on
October 24.38 and tracked by the camera continuously until sunrise
four hours later.  During this time the comet brightened to visual
magnitude 3.5. Comet Holmes was next observed just after sunset on
October 25.23 at visual magnitude 2.5 where it remained
approximately constant over the next three days. The comet then
began to dim slowly and was followed into the early months of 2008
with periods of dense time coverage.
\end{abstract}


%%%%%%%%%%%%%%%%%%%%%%%%%%%%%%%%%%%%%%%%%%%%%%%%%%%%%%%%%%%%%%%%%%%%%%%%
\section{Introduction}

Comets have been noted in the sky for almost as long as history
has been recorded.  A verifiable light curve for any bright comet
is rare, however, since human observers are hard to calibrate,
photographic magnitudes can be unreliable for bright objects, and
because telescopes with modern CCDs typically have fields of view
too small to contain a comet that has reached naked-eye
visibility.  Even for comets monitored by CCD, unusual cometary
events are rarely recorded, since observers are typically alerted
to look only after such an event has occurred.

In recent years, however, a class of all-sky cameras has begun to
be used routinely in astronomy, primarily to assess sky conditions
(Nemiroff &  Rafert 1999; Pickering 2006; Shamir & Nemiroff 2005;
Shamir & Nemiroff 2005). These cameras typically utilize fisheye
lenses with fields of view in excess of 150� and can thus capture
images of a comet of almost any size.  Furthermore, these cameras
typically operate every clear night, and so are likely to be
operating during an unusual cometary outburst.

In 2007 October, periodic comet 17P Holmes underwent an unusual
outburst which increased its brightness from an apparent visual
magnitude of about 17 to near 3.  This paper reports details from
an all-sky camera that captured this cometary outburst and
monitored the brightness of the comet continuously for the next
few hours and for many nights over the next three months.

%%%%%%%%%%%%%%%%%%%%%%%%%%%%%%%%%%%%%%%%%%%%%%%%%%%%%%%%%%%%%%%%%%%%%%%%
\section{A Brief History of Comet Holmes}

Comet Holmes was discovered by Edwin Holmes (London, England) on
November 6, 1892 within the Andromeda Galaxy (M31). The comet
brightened from dimmer than visual magnitude 17 to about visual
magnitude 2.8 over about 42 hours, creating a coma about 5 arc
minutes diameter (Whipple 1984).  The comet's discovery was
confirmed by Edward Walter of the Royal Observatory in Greenwich,
England (Whipple 1984; Bobrovnikoff 1943).  Unexpectedly, Comet
Holmes underwent a second outburst only few months later, on
January 16, 1893.  In the second outburst, the comet brightened to
visual magnitude 8 and exhibited a coma of 41 arc seconds in
diameter.  The comet steadily exhibited a larger coma until late
the next night and then steadily faded after the next outburst.
The last observation before fading from visibility was made by H.
C. Wilson of Goodsell Observatory in Northfield, Minnesota on
April 4, 1893 (Bobrovnikoff 1943; Zwiers 1912). The comet was lost
after 1906 until being re-acquired on July 16, 1964 by Elizabeth
Roemer of the Naval Observatory in Arizona, USA (Whipple 1984).

Several attempts have been made to determine the comet's orbital
elements. The first orbit determination was calculated by H. C.
Kreutz using three positions measured on November 9, 10, and 11,
1892 (Zwiers 1912).  Kreutz discussed the difficulty of
calculating Comet Holmes' orbit and introduced four potential
parabolic orbits satisfying three observations with perihelion
passage time ranging from February 28 - June 7, 1892 and an
orbital period of 6.9 years. During the next few weeks, several
more attempts were made to more precisely determine Comet Holmes'
orbit.  These attempts also derived rather different orbital
elements with perihelion passage estimates ranging from February
28 to August 16, 1892.

%%%%%%%%%%%%%%%%%%%%%%%%%%%%%%%%%%%%%%%%%%%%%%%%%%%%%%%%%%%%%%%%%%%%%%%%
\section{The Outburst and the Light Curve of Comet Holmes in 2007-2008}

On October 24, 2007, Comet Holmes underwent an outburst similar to
its first outburst.  During the early hours of October 24, 2007,
the comet became much brighter, increasing its brightness from a
visual magnitude of about 18 to 2.5 over less than two days. Comet
Holmes became the third brightest object in the constellation of
Perseus (see Figure~\ref{fig1}-b) and was visible to the unaided
eyes of even casual observers.

%%%%%%%%%%%%%%%%%%%%%%%%%%%%%%%%%%%%%%%%%%%%%%%%%%%%%%%%%%%%%%%%%%%%%%%%

\begin{figure}[t]
\label{fig1}
\centering
\includegraphics[scale=0.35, angle=270]{1ab1.eps} \caption{Two images extracted from MMTO All Sky Camera image archive.  Comet Holmes is barely visible on the image on the left, taken on October 16 when it was fainter than 17th magnitude, while only 18 nights later, at the same sidereal time, the comet is easily visible on the image on the right at a visual magnitude of 2.8.}
\end{figure}
\\
Although hampered by moonlight, the MMT All-Sky Camera was able to
capture Comet Holmes on the night of its sudden brightening.
Through the course of its normal operation, the All-Sky Camera was
then able to follow the evolution of Comet Holmes for several
months after its outburst. On a dark night, the system defaults to
an 8.533 second exposure time which results in a limiting
magnitude of about 5.5 in V and 6 in R. Sensitivity is decreased
by moonlight, however, due to glare, reduced gain, and reduced
exposure time.

It was necessary to divide Comet Holmes' light curve into two
graphs according to our continuous observations of the comet from
the time it became visible on MMTO All Sky Camera images over the
next three months. Figure~\ref{fig2} shows a plot of the visual
magnitude of Comet Holmes over the first three days where the
comet exhibited extreme magnitude change from below visibility to
magnitude 2.5 in less than 24 hours. Data points are plotted for
every hour by averaging over each 10-second exposure. While
Figure~\ref{fig3} shows a plot of the comet magnitude over the
next three months where the comet exhibits steady fading phase.
Data points are plotted for every notable magnitude change, nearly
for every 24 hours. Magnitude estimates in both graphs were made
by comparison to stars of cataloged magnitudes (Henry Draper
Catalogue at Harvard College Observatory).

\begin{figure*}[!ht]
\label{fig2} \centering
\includegraphics[scale=0.50, angle=0]{2.eps} \caption{The Comet Holmes' light curve from October 24-26, 2007}
\includegraphics[scale=0.50, angle=0]{3.eps}
\caption{The Comet Holmes' light curve from October 27, 2007-
January 1, 2008} \label{fig3}
\end{figure*}

Figure~\ref{fig2} shows that Comet Holmes started its outburst at
the early hours of October 24, 2007, in particular about 9.5 am UT
(Universal Time) and it was of magnitude 5.5 to be visible to
unaided eyes. In less than 24 hours, comet 17P surprisingly
brightened to magnitude 3.5 that is enough to be seen in full-moon
night and town light pollution. In the next night, October 25,
2007, at 06:53 am UT, the comet reached magnitude 2.5. This is
followed by complete constancy at maximum light sustained for 2
days after the comet outburst.

Figure~\ref{fig3} shows that Comet Holmes remained visible over
the next three months. Night to night, from October 27, 2007-
January 1, 2008, Comet 17P is slowly and steadily fading through
periods of about three days long.


%%%%%%%%%%%%%%%%%%%%%%%%%%%%%%%%%%%%%%%%%%%%%%%%%%%%%%%%%%%%%%%%%%%%%%%%
\section{Quantifying Comet Holmes' Light Curve}

For explanatory and future predictive value, an attempt was made
to quantify the evolution of the brightness of Comet Holmes as a
function of time.  Comet for Windows software (Seiichi Yoshida,
1995-2004) was used to analyze the brightness measurements  of
comet Holmes and optimize a light curve as shown in
Figure~\ref{fig4}.

\begin{figure*}[!ht]
\label{fig4} \centering
\includegraphics[scale=0.55, angle=0]{41.eps} \caption{Comet Holmes light curve according to magnitude
estimations is drawn by dots and the solid line represents the
calculated magnitude formulas for our magnitude estimations.}
\end{figure*}

The light curve is calculated from all brightness measurements as
divided into two periods, starting after Comet Holmes flare on
Oct. 24, 2007.  The magnitude formula, which represents magnitude
change over 23 days after the flare, is

\begin{equation}
m_1 = -15.69 + 5 \log d + 43.6 \log r
\end{equation}

Subsequently, (from Nov. 16, 2007 - Jan. 1, 2008) the comet slowly and steadily fading according to

\begin{equation}
m_2 = -1.63 + 5 \log d + 8.71 \log r
\end{equation}
where $m$ is the comet's apparent bolometric magnitude; and $d$
and $r$ are its geocentric and heliocentric distances
respectively.

%%%%%%%%%%%%%%%%%%%%%%%%%%%%%%%%%%%%%%%%%%%%%%%%%%%%%%%%%%%%%%%%%%%%%%%%
\section{Summary and Conclusions}


Comet outbursts were discovered many decades ago (Bobrovnikoff
1943 and Zwiers 1912), though a complete picture of such odd
occurrences is not resolved yet. Comet light curves provide
important information about comet flares (see, for example,
Churyumov and Filoneko 1993). In an attempt to provide a clearer
picture about this unusual cometary flare, our paper summarizes
and analyzes the results of observations of Comet 17P/Holmes'
flare that began in October 2008.  These observations were taken
with the MMTO All Sky Camera over three months from the beginning
of its outburst.  From the time Comet Holmes became visible to the
MMTO All Sky Camera, it brightened from visual magnitude 5.5 to
magnitude 2.5 in less than 24 hours, and then brightened only
slightly over the next two days.  To the best of our knowledge, no
similar comet brightening has ever been recorded in such detail by
a single dedicated instrument before. After reaching its peak,
Comet Holmes steadily faded to fainter magnitudes over the next
three months to reach magnitude 3.7 on January 1, 2008.  Again, to
the best of our knowledge, no comet has ever been monitored by a
single dedicated instrument for so long a period of time
previously.  It is our hope that the Comet Holmes' light-curves
presented here will provide useful constraints for future comet
outburst models.

%Old coclusion
%Comet outbursts were discovered many decades ago, though a
%complete picture of such odd occurrences is not resolved yet.
%Comet light curves provide important information about comet
%flares. In an attempt to provide a clearer picture about this
%unusual cometary flare, our paper summarizes and analyzes the
%results of observations of Comet 17P/Holmes' flare that began in
%October 2008.  These observations were taken with the MMTO All Sky
%Camera over three months from the beginning of its outburst.  From
%the time Comet Holmes became visible to the MMTO All Sky Camera,
%it brightened from visual magnitude 5.5 to magnitude 2.5 in less
%than 24 hours, and then brightened only slightly over the next two
%days.  Afterwards, Comet Holmes steadily faded to fainter
%magnitudes over the next three months to reach magnitude 3.7 on
%January 1, 2008.  It is our hope that the Comet Holmes'
%light-curves presented here will yield significant results for
%further attempts to correlate the comet apparent bolometric
%magnitude with determination to the unsteady-states processes in
%comets in general.


%%%%%%%%%%%%%%%%%%%%%%%%%%%%%%%%%%%%%%%%%%%%%%%%%%%%%%%%%%%%%%%%%%%%%%%%
\acknowledgments I gratefully acknowledge support of my travel to
the US by the International Astronomical Union (IAU) through the
Exchange of Astronomers Programme.

%%%%%%%%%%%%%%%%%%%%%%%%%%%%%%%%%%%%%%%%%%%%%%%%%%%%%%%%%%%%%%%%%%%%%%%%
\section{References}

%Use \cite! command to cite reference(s).
%\cite{bag02}  --->  Author (year)
%\citep{bag02} --->  (Author year)
%\citet{bag02} --->  (Author year; Author year)
%\citeauthor{ale94}
%\citeyear{ale94}


\bibitem[Bobrovnikoff(1943)]{bob43} Bobrovnikoff, N. T. " The Periodic Comet Holmes (1892 III)," PA, 51, 542B, 1943.

\bibitem[Churyumov(1993)]{Chu93} Churyumov, K. I. and Filoneko, V. S., "Phase Dependencies of Cometary Light Curves," Abstracts of IAU Symp. 160: Asteroids, Comets, Meteors, Belgirate (Novara), Italy, , P. 66, June 14-18, 1993.

\bibitem[Seiichi(1995-2004)]{Com04} Comet for Windows software by Seiichi Yoshida, 1995-2004.

\bibitem[Henry()]{Hen} Henry Draper Catalogue, was compiled by Annie Jump Cannon and co-workers at Harvard College Observatory under the supervision of Edward C.
Pickering.

\bibitem[Nemiroff(1999)]{Nem99} Nemiroff, R. J. and Rafert, J. B., "Toward a Continuous Record of the Sky," PASP, 111, 886-897, July 1999.

\bibitem[Osamu()]{Osa} Osamu Ajiki, "Orbit Viewer applet (AstroArts)", and further modified by Ron Baalke (JPL). http://www.nasa.gov/

\bibitem[Pickering(2006)]{Pic06} Pickering, T. E., "The MMT All-Sky Camera," SPIE 6267, June 2006.

\bibitem[Shamir(2005b)]{Sha05b} Shamir, L. and Nemiroff, R. J., "All-Sky Relative Opacity Mapping Using Nighttime Panoramic Images," PASP, 117, 835, 972-977, September 2005.

\bibitem[Shamir(2005a)]{Sha05a} Shamir, L. and Nemiroff, R. J., "PHOTZIP: A Lossy FITS Image Compression Algorithm that Protects User Defined Levels of Photometric Integrity," 129, 1, 539-546, Jan 2005.

\bibitem[Whipple(1984)]{whi84} Whipple, F. L., "Comet P/Holmes, 1892III: A case of duplicity?" Icarus, 60, issue 3, 1984.

\bibitem[Zwiers(1912)]{Zwi12} Zwiers, H. J., "Researches on the orbit of the periodic comet Holmes and on the perturbations of its elliptic motion", De Roever Kröber & Bakels, 1912.




%\nocite{*}
\bibliographystyle{spr-mp-nameyear-cnd}
%\bibliography{myref}
\bibliography{biblio-u1}

\end{document}

%%
%% End of file `sample.tex'.
